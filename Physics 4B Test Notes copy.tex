
\documentclass[english]{article}
\usepackage[T1]{fontenc}
\usepackage[latin9]{inputenc}
\usepackage{geometry}
\geometry{verbose,tmargin=1cm,bmargin=1cm,lmargin=1cm,rmargin=1cm}
\usepackage{amsmath}
\usepackage{esint}
\usepackage{babel}
\begin{document}
\emph{Electric Potential Energy:} $\varDelta U=U_{f}-U_{i}=-W$

\emph{Electric Potential: }$V=\frac{U}{q}=\frac{kq}{r}$

Electric potential energy per unit charge at a point in an electric
field is called electric potential

\emph{Electric Potential Difference:} $\varDelta V=V_{f}-V_{i}=\frac{U_{f}}{q}-\frac{U_{i}}{q}=\frac{\varDelta U}{q}=-\frac{W}{q}=-\frac{W_{\infty}}{q}$

\emph{Electric Potential from a Electric Field:} $V{}_{f}-V_{i}=-\intop\overrightarrow{E}\bullet\overrightarrow{ds}=EA=V$

Potential increases in the direction pointing away from the electric
field vector

\emph{Potential Due to Point Charge:} $V=\frac{1}{4\pi\varepsilon_{0}}\frac{q}{r}$

A positively charged particle produces a positive electric potential.
A negatively charged particle produces a negative electric potential.

\emph{Electric Potential of a Ring Charge Q: $V_{ring}=\frac{1}{4\pi\varepsilon_{0}}\frac{Q}{\sqrt{R^{2}+z^{2}}}$}

\emph{Potential Charge of a Disk: $V_{disk}=\frac{Q}{2\pi\varepsilon_{0}R^{2}}(\sqrt{R^{2}+z^{2}}-|z|$}

Relating electric potential to electric fields

Equipotential lines are perpendicular to electric field and get further
apart as distance increase from point charge

Electric Potential is a Scalar

\emph{Electric dipole:} $V=\frac{1}{4\pi\varepsilon_{0}}\frac{pcos\theta}{r^{2}}$

\emph{Capacitance:} The ability to store charge: $q=CV$or $\frac{q}{C}=V$
or $\frac{q}{V}=C$ (farad)

The greater the capacitance, the more charge required

$q=\varepsilon_{0}EA$ (electric field)

$E_{capcitor}=\sigma/\varepsilon_{0}=Q/\varepsilon_{0}A$

$V=Ed$

\emph{Parallel Plate Capacitor:} $C=\frac{\varepsilon_{0}A}{d}=\frac{Q}{V}$
and $E=\frac{Q}{\varepsilon_{0}A}$

Capacitance depends only on geometric factors, the plate area A and
separation d

\emph{Cylindrical Capacitor:} $C=2\pi\varepsilon_{0}\frac{L}{ln(b/a)}$(Based
on length and two radii)

Units of $\varepsilon_{0}=8.885x10^{-12}F/m$ 

\emph{Spherical Capacitor: }$C=4\pi\varepsilon_{0}\frac{ab}{b-a}$

\emph{Isolated Sphere:} $C=4\pi\varepsilon_{0}R$

\emph{Capacitors in Parallel:} In parallel, $\varDelta V$across the
capacitors is the same, and the total charge stored on the capacitors
is the sum of the charges stored on all the capacitors. 

$q_{1}=C_{1}V$, $q_{2}=C_{2}V$

So the total becomes:$q=q_{1}=q_{2}=(C_{1}+C_{2})V$

So, $C_{eq}=\frac{q}{V}=C_{1}+C_{2}$

-Same V, different charges on plates

\emph{Capacitors in Series:} When capacitors are in series, they have
the same charge, and the sum of the potential differences is equal
to the applied V

$V_{1}=\frac{q}{C_{1}},V_{2}=\frac{q}{C_{2}}$$\Rightarrow V=V_{1}+V_{2}=q(\frac{1}{C_{1}}+\frac{1}{C_{2}})$and
$C_{eq}=\frac{q}{V}=\frac{1}{\frac{1}{C_{1}}+\frac{1}{C_{2}}}$$=(\frac{1}{C_{1}}+\frac{1}{C_{2}})^{-1}$

-Same charge q, different V across

\emph{Potential Energy stored in an electric field/capacitor: }$U_{c}=\frac{q^{2}}{2C}=\frac{1}{2}CV^{2}=\frac{1}{2}QV$

\emph{Capacitor with a dielectric:} $E=\frac{1}{4\pi\kappa\sum\varepsilon_{0}}\frac{q}{r^{2}}$
or $E=\frac{\sigma}{\kappa\varepsilon_{0}}$

$C=\kappa\varepsilon_{0}\frac{A}{d}=\frac{Q}{V}=\frac{Q_{0}}{V/\kappa}=\kappa C$

\emph{Potential across a capacitor with a dielectric: }$V_{c}=\frac{V_{0}}{\kappa}$

If the battery remains connected, voltaged wont change, and if the
batery is removed the total charge in the system must remain the same

\emph{Dielectric Constant $\kappa$: }$\kappa=\frac{C}{C_{0}}$

\emph{Permittivity of $\kappa$=} $\kappa\varepsilon_{0}$

\emph{Current:} The rate at which charge flows: $i=\int\overrightarrow{J}\cdot d\overrightarrow{A}=\frac{dq}{dt}=JA=\frac{\varDelta q}{\varDelta t}=\frac{nALe}{L/\vec{v_{d}}}=nAe\overrightarrow{v_{d}}=\frac{V}{R}$,
$\Delta t=\frac{L}{\overrightarrow{v_{d}}}$

\emph{Units of Current:} 1 ampere = $C/s$

\emph{Current Density:} $J=\frac{i}{A}=(ne)\overrightarrow{v_{d}}$,
if too high can lead to resitive heating and melted wires

\emph{Conductiviy: $J=\sigma E$}

\emph{Drift Speed:} $\overrightarrow{v_{d}}=\frac{i}{nAe}=\frac{J}{ne}$

\emph{Resistance:} $R=\frac{V}{i}=\rho\frac{L}{A}$

\emph{Resistivity:} $\rho=\frac{E}{J}=\frac{1}{\sigma}$ or $\overrightarrow{E}=\rho\overrightarrow{J}$
or $\vec{J}=\sigma\overrightarrow{E}$

Resistance is a property of an object, resistivity is a property of
a material

$E=\frac{V}{L}=\rho\overrightarrow{J}$and $J=\frac{i}{A}$ and $\rho=\frac{E}{J}=\frac{V/L}{i/A}$

\emph{Ohm's Law} is an assertion that the current through a device
is always directly proportional to the potential difference applied
to the device. A conducting device obeys Ohm\textquoteright s law
when the resistance of the device is independent of the magnitude
and polarity of the applied potential difference. A conducting material
obeys Ohm\textquoteright s law when the resistivity of the material
is independent of the magnitude and direction of the applied electric
field.

\emph{Power:} $P=\frac{dU}{dt}=V\frac{dq}{dt}=iV=i^{2}R=\frac{V^{2}}{R}$,
units J/s = W = Watts (amperes times volts)

\emph{Electromotive Force}: $\epsilon=\frac{dW}{dq}=iR$, $i=\frac{\epsilon}{R}$

\emph{Loop Rule:} The algebraic sum of the charges in potential encountered
in a complete traversal of any loop of a circuit must be zero

\emph{Resistance Rule:} For a move through resistance in the directions
of the current, the change in potential is -iR; in the opposite direction
it is +iR

\emph{EMF Rule:} For a move through an ideal emf device in the direction
of the emf arrow, the change in potential is +$\epsilon$ and in the
opposite direction it is -$\epsilon$

\emph{Resistances in Series:} When a potential difference V is applied
across resistances connected in series the resistances have identical
current, the sum of the potential differences across the resistances
is equal to V:$\epsilon-iR_{1}-iR_{2}=0$ so $i=\frac{\epsilon}{R_{1}+R_{2}}$
or $i=\frac{\epsilon}{R_{eq}}$ or $R_{eq}=\sum R_{k}$

-Same i, Shared $\Delta V$

\emph{Resistances in Parallel:} When a potential difference V is applied
across resistances connected in parallel, the resistances all have
the same potential V. Resistances connected in parallel can be replaced
with an equivalent resistance $R{}_{eq}$ that has the same V and
same total current i as the actual resistances

$i=\frac{V}{R_{1}},i_{2}=\frac{V}{R_{2}}$ so $i=i_{1}+i_{2}=V(\frac{1}{R_{1}}+\frac{1}{R_{2}})$
so $i=\frac{V}{R_{eq}}$ so $\frac{1}{R_{eq}}=\sum\frac{1}{R_{n}}$

-Same $\varDelta V$

\emph{Series Circuit: }Electric current is the same through all devices.
The total voltage divides among the individual electrical devices
in the circuit If one device fails, current in the entire circuit
ceases.

\emph{Parallel Circuit: }Forms branches, each of which is a separate
path for the flow of electrons Voltage is the same across each device.
The total current in the circuit equals the sum of the currents in
its parallel branches.

\emph{Potential Difference Between Two Points:} To find the potential
between any two points in a circuit, start at one point and traverse
the circuit to the other point, following any path, and add algebraically
the changes in potential you encounter

\emph{Potential Across Real Battery: }$\varDelta V=\epsilon-ir$

\emph{Power of an emf device: }$\sum$$P_{emf}=i\epsilon$

\emph{Junction Rule:} The sum of the currents entering any junction
must be equal to the sum of the currents leaving that junction $i_{1}=i_{2}+i_{3}$

\emph{RC Circuits}: When an emf is applied to a resistance R and capacitance
C in a series with a switch, the charge on the capacitor increases
according to: $q=C\epsilon(1-e^{-t/RC}$(Charging a capacitor), in
which $C\epsilon=q_{0}$ is the equilibrium and final charge and $RC=\tau$
is the capacitive time constant of the circuit. During the charging,
the current is: $i=\frac{dq}{dt}=(\frac{\epsilon}{R})e^{-t/RC}$.
When a capacitor discharges through the resistance R, the charge on
the capacitor decays according to $q=q_{0}e^{-t/RC}$(discharging
a capacitor) and during discharging the current is $i=\frac{dq}{dt}=-(\frac{q_{0}}{RC})e^{-t/RC}$(discharging
a capacitor)
\end{document}
