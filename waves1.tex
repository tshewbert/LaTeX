%% LyX 2.1.1 created this file.  For more info, see http://www.lyx.org/.
%% Do not edit unless you really know what you are doing.
\documentclass[english]{article}
\usepackage[T1]{fontenc}
\usepackage[utf8]{luainputenc}

\makeatletter
%%%%%%%%%%%%%%%%%%%%%%%%%%%%%% User specified LaTeX commands.
\renewcommand\[{\begin{equation}}
\renewcommand\]{\end{equation}}

\makeatother

\usepackage{babel}
\begin{document}

\title{Waves, Part One}

\maketitle

\subsubsection*{Types of Waves}
\begin{enumerate}
\item Mechanical Waves: These are the most common waves and include water
waves, seismic waves and sound waves. They are governed by Newton's
Laws and much exist within a material medium.
\item Electromagnetic Waves: These waves include x-rays, ultraviolet light,
radio, microwaves and radar. These waves do not require a medium to
travel through to exist. EM waves travel through a vacuum at the speed
of light.
\item Matter Waves: These waves are associated with protons, neutrons, electron,
and other fundamental particles.
\end{enumerate}

\subsubsection*{Transverse and Longitudinal Waves}

A wave can be sent along a stretched, taut string is a a simple form
of a mechanical wave. A pulse will travel along the string. As this
pulse distorts the string's shape the pulse moves along the string
at some velocity $\overrightarrow{v}.$ If a continuous wave travels
down the string at $\overrightarrow{v}$ in continuous harmonic motion,
the motion that causes the pulse can be described by a sinusoidal
function of time. In an ideal string, no friction-like forces within
the string will cause the wave to die out as it travels along. In
an ideal string, it is also assumed that we need not consider the
wave rebounding from the opposite end from which it began.

Monitoring wave forms is one way to study waves. Another option is
to monitor an element of string as it oscillates up and down as a
wave passes it. The displacement of the the up and down oscillations
in orthogonal to the direction of travel of the wave. This motion
is said to be \emph{transverse}, and the wave is referred to as a
\emph{transverse wave.} 

For a \emph{longitudinal wave}, the example of pushing and pulling
on a piston filled with air can be used. The motion of the air is
parallel to the direction of the wave's travel is referred to as \emph{longitudinal.}

Both of these types of waves are \emph{traveling waves}, since they
travel from one point to another. It is the wave that moves from end
to end, not the material through which the wave moves.


\subsubsection*{Wavelength and Frequency}

To describe the wave on a string a function that gives the shape of
the wave is needed. This means a relation in the form:

\[
y=h(x,t)
\]


Where \emph{y }is the transverse displacement of any string element
described by the function \emph{h, }where \emph{t }is time and \emph{x
}is the position of the element along the string. In general the shape
is sinusoidal, then the sine or cosine function can be used to describe
the shape. Here the sine function will be used:

\[
y(x,t)=y_{m}\sin(kx-\omega t)
\]


This equation is written in terms of \emph{x,} meaning that it can
be used to find the displacements of all the elements with respect
to time, \emph{t. }

$y_{m}$is the amplitude of a wave, the maximum displacement of the
wave elements from their equilibrium positions as the wave passes
through them. Since it is magnitude, amplitude can always be considered
a positive quantity.

The phase is described by $kx-\omega t$. As the waves sweeps through
an element of the string linearly through a particularly position,
the phases changes with respect to time \emph{t. }Therefore, sine
also changes, alternating between 1 and -1. Theses max and min values
of the sine function correspond to the peaks and valleys of $y_{m}$.
The amplitude of the wave determines the extremes of the elements
displacement.


\subsubsection*{Wavelength and Angular Wave Number}

The wavelength, $\lambda$,of a wave is parallel to the direction
of the wave's travel and describes the distances between the shape
of the waves. The equation that describes the wavelength can be written:
\[
y(x,0)=y_{m}\sin kx
\]


The displacement \emph{y} at both ends of the wavelength must be the
same, so $x=x_{1}$and $x=x_{1}+\lambda$. This allows the previous
equation to be rewritten as:

\begin{eqnarray*}
y_{m} & = & \sin k(x_{1}+\lambda)\\
 & = & \sin(kx_{1}+k\lambda)
\end{eqnarray*}


Since a sine function repeats itself every $2\pi$, $k\lambda=2\pi$
or $k=\frac{2\pi}{\lambda}$, where \emph{k }is the angular wave number
of the wave, and its units are rad/m, or the inverse meter.


\subsubsection*{Period, Angular Frequency and Frequency}

If a single element of the string is monitored, it will be seen that
at that position \emph{x }the string moves up and down in simple harmonic
motion. If $x=0$, then $y(0,t)=y_{m}\sin(k(0)-\omega t)$ and then
$-y_{m}\sin(\omega t)$. If this is graphed, it will show the displacement
versus time, not the shape of the wave.

A \emph{period} of oscillation \emph{T }is defined at the time any
string element takes to move through one full oscillation. To find
the angular frequency $\omega$ or period \emph{T, }we use $-y_{m}\sin(\omega t)$:

\begin{eqnarray*}
-y_{m}\sin(\omega t) & = & -y_{m}\sin\omega(t_{1}+T)\\
 & = & -y_{m}\sin(\omega t_{1}+\omega T)
\end{eqnarray*}


This is only true if $2\pi=\omega T$. The can be written to describe
\emph{angular frequency:}

\[
\omega=\frac{2\pi}{T}
\]


The SI unit of angular frequency is radians per second.

The frequency \emph{f }of a wave is defined as $1/T$ and is related
to angular frequency by:
\[
f=\frac{1}{T}=\frac{\omega}{2\pi}
\]



\subsubsection*{Phase Constant}

The phase constant $\phi$ can be used to generalize the equation
$y(x,t)=y_{m}\sin(kx-\omega t)$ by transforming it to $y(x,t)=y_{m}\sin(kx-\omega t+\phi)$.
The value of $\phi$ can be chosen so the function gives a different
displacement and slope at $x=0$ and $t=0$. Basically, the function
is shifted by the value of $\phi$.


\subsubsection*{The Speed of a Traveling Wave}

If a wave is traveling in the positive \emph{x }direction over a period
of time, this can be described by $\Delta x/\Delta t$. At an instantaneous
point, this can be written in the differential form $\frac{dx}{dt}$,
which is the wave speed\emph{ v.}

If the displacement \emph{y }remains the same, then $kx-wt$ is a
constant, but \emph{x }and \emph{t }are both changing, which means
that as time increases, so does displacement to keep the argument
constant, confirming that the wave pattern in moving in a positive
direction.

To find the wave speed, we can take the derivative of $kx-wt$:

\begin{eqnarray*}
k\frac{dx}{dt}-\omega & = & 0\\
\frac{dx}{dt}= & v & =\frac{\omega}{k}
\end{eqnarray*}


Assuming that $k=2\pi/\lambda$and $\omega=2\pi/T$, these equations
can be rewritten as:

\[
v=\frac{\omega}{k}=\frac{\lambda}{T}=\lambda f
\]


Which describes wave speed traveling in the positive direction, \emph{x.}

For a wave speed in the negative \emph{x }direction:

\[
\frac{dx}{dt}=-\frac{\omega}{k}
\]



\subsubsection*{Wave Speed on a Stretched String}

The properties of the medium ultimately set the speed of a wave even
though the speed is relation to the wave's frequency and wavelength.
For a wave to travel through a medium, it must cause the particles
of the medium to oscillate as it passes, which requires mass and elasticity,
for kinetic and potential energy. 

We can use dimensional analysis to examine the mass and elasticity
to find a speed \emph{v, }which has the dimension of length divided
by time. The value of mass is the mass of the string element, which
is the mass \emph{m }of the string divided by the length \emph{l}.
This is the linear density $\mu$of the string.

A wave cannot be sent along a string unless the string is under tension.
This tension $\tau$ is being caused by forces on either end of the
string stretching and pulling taut the string. As the wave travels
along the string, it displaces the string by causing stretching, with
adjacent sections of string pulling on each other because of tension.
This tension in the string can be associated with the elasticity of
the string. This can be described by $MLT^{-2}$.

Combining $\mu$ with $\tau$ to get \emph{v} ends up with:

\[
v=C\sqrt{\frac{\tau}{\mu}}=C\sqrt{\frac{MLT^{-2}}{ML^{-1}}}
\]


Where \emph{C }is a dimensionless constant that cannot be determined
through dimensional analysis.

It is also possible to derive wave speed from Newton's Second Law.
To do this, we must consider and single, symmetrical pulse that is
moving in the positive \emph{x }direction with a speed \emph{v. }Choosing
a reference point where the pulse remains stationary, the string appearing
to be moving, not the wave. A small string element $\Delta l$ within
the pulse, an arc of a circle \emph{R, }and an angle $2\theta$, are
all parts. A force $\vec{\tau}$with a magnitude equal to the tension
in the string pulls on the string at each end. This allows the horizontal
components of the forces to cancel, but allows the vertical component
to add forming a radial restoring force $\vec{F}$:

\[
F=2(\tau\sin\theta)\approx\tau(2\theta)=\tau\frac{\Delta l}{R}
\]


The mass of this element is:
\[
\Delta m=\mu\Delta l
\]


Since it is moving in an arc, we can use centripetal acceleration
towards the center of that circle, and transforming Newton's Second
Law gives us the form:

\[
\frac{\tau\Delta l}{R}=(\mu\Delta l)\frac{v^{2}}{R}
\]


Solving for \emph{v:}
\[
v=\sqrt{\frac{\tau}{\mu}}
\]


The speed of a wave along a stretched ideal string is dependent only
on the tension and linear density of the string and not the frequency
of the wave.


\subsubsection*{The Energy and Power of a Wave Traveling Along a String}

As a wave moves on a stretched spring, it provides transports both
kinetic and potential energy.

\emph{For Kinetic Energy (KE): }The transverse velocity $\vec{\mu}$is
associated with kinetic energy. A string element of mass $dm$ oscillates
transversely with simple harmonic motion as the wave passes. When
the wave passes through $y=0$, KE is at its maximum, and likewise,
when the wave passes through $y=y_{m}$KE is at its minimum.

\emph{Elastic Potential Energy(PE): }Elastic potential energy is associated
with the changes in length, $dx$, of the a string element as it oscillates
transversely. At $y=0$, the string has its maximum stretch and therefore
the maximum potential energy, and likewise at $y=y_{m}$ the value
of $dx$ is undisturbed therefore its PE is zero.

The maximum PE and KE are both at $y=0$


\subsubsection*{The Rate of Energy Transmission}

The kinetic energy differential $dK$ is related to a string element
mass differential $dm$ by
\[
dK=\frac{1}{2}dmu^{2}
\]


where $u$ is the transverse speed of the string element that is oscillating.
The value of $u$ can be found by taking the partial derivative $\frac{\partial y}{\partial t}$
of $y(x,t)=y_{m}\sin(kx-\omega t)$ which leads to
\[
u=-\omega y_{m}\cos(kx-\omega t)
\]


This can be rewritten using the relation $dm=\mu dx$
\[
dK=\frac{1}{2}(\mu dx)(-\omega y_{m})^{2}\cos^{2}(kx-\omega t)
\]


Divide that by the time differential $dt$ gives the rate which kinetic
energy is carried along by the wave
\[
(\frac{dK}{dt})=\frac{1}{2}\mu\frac{dx}{dt}\omega^{2}y_{m}{}^{2}\cos^{2}(kx-\omega t)
\]


Which can also be described at the average rate KE is transported,
and can be simplified to
\[
\frac{dK}{dt}_{avg}=\frac{1}{4}\mu v\omega^{2}y_{m}^{2}
\]


Using this equation, the average power can be described as
\[
P_{avg}=2(\frac{dK}{dt})_{avg}=\frac{1}{2}\mu v\omega^{2}y_{m}^{2}
\]



\subsubsection*{The Wave Equation}

Using Newton's Second Law to the an element on a stretched string
with a wave passing through it, a general differential equation, referred
to as the wave equation, can be derived that describes the travel
of waves of any type.

If it is assumed that the wave amplitude is so small that the element
is only slightly tilted from the $x$ axis, then the force $\vec{F_{2}}$
on the right end of the element has a magnitude equal to $\tau$ and
is directed slightly upward. Likewise, $\vec{F_{1}}$on the left end
has a magnitude equal to $\tau$ but is directed downward. Because
of the slight curvature, there is an upward acceleration produced,
$a_{y}$. If mass is differential, then Newton's second law can be
rewritten
\[
F_{2y}-F_{1y}=dm\, a_{y}
\]


Mass $dm$ can be rewritten as $dm=\mu l$ and further rewritten as
$dm=\mu dx$. Acceleration can be rewritten as the second derivative
of\emph{ y} with respect to \emph{t, $\frac{d^{2}y}{dt^{2}}$. }The
forces can be described as the string slope $\frac{F_{2y}}{F_{2x}}=S_{2}$and
$\tau=F_{2x}$ because the element is only slightly tilted, and can
be rewritten as $F_{2y}=\tau S_{2}$and through similar analysis $F_{1y}=\tau S_{1}$.
Since the string element is short, the slope can actually be described
by $\frac{dy}{dx}$. Through some substitution and simplifying we
can reduced the equation $F_{2y}-F_{1y}=dm\, a_{y}$ to

\[
\frac{\partial^{2}y}{\partial x^{2}}=\frac{1}{v^{2}}\frac{\partial^{2}y}{\partial t^{2}}
\]


using partial derivatives to differentiate on the left with respect
to \emph{x }and on the right with respect to \emph{y.}

This is the general differential equation that describes the travel
of waves of all types.


\subsubsection*{The Principle of the Superposition of Waves}

Overlapping waves add algebraically to produce a resultant wave, and
overlapping waves do not in any way alter the travel of each other,
which can be described by the equation
\[
y'(x,t)=y_{1}(x,t)+y_{2}(x,t)
\]



\subsubsection*{Interference of Waves}

The resultant wave described by the previous equation depends on the
extent to which the waves are in phase. Combining waves is called
interference. If two waves of the same amplitude and wavelength travel
in the same direction along a stretched string, they will interfere
two produce a resultant wave, if both the waves are sinusoidal. This
result can be described by the equation
\[
y'(x,t)=[2y_{m}\cos\frac{1}{2}\phi]\sin(kx-\omega t+\frac{1}{2}\phi
\]


There are two differences between the resultant wave and the interfering
waves. First, its phase is $\frac{1}{2}\phi$ and second, its amplitude
$y'm$ is the magnitude of the quantity
\[
y'_{m}=|2y_{m}\cos\frac{1}{2}\phi|
\]


If $\phi=0$, then the interfering waves are exactly in phase. Fully
constructive interference is the interference that produces the greatest
possible amplitude. 

If $\phi=\pi$, then the interfering waves are exactly out of phase.
Even though two waves were sent along the string, there is no motion
on the string. The is referred to as fully destructive interference.


\subsubsection*{Phasors}

A \emph{phasor} can be used to represent a string wave vectorially.
A phasor is a vector that has the same magnitude as the amplitude
of the wave and rotates about the origin. By this, the angular speed
is equal to the angular frequency $\omega$. Phasors can be used to
combined waves even if the waves have different amplitudes. Using
vectors to add the phasors component wise allows this to happen, resulting
in a new $y_{m}$ which is the magnitude of the vector addition of
the horizontal and vertical components.


\subsubsection*{Standing Waves}

If two waves are traveling in opposite directions, the resultant wave
can be found by applying the superposition principle. If two waves
of the same amplitude and wavelength travel in opposite directions,
their interference with each other will produce a standing wave. The
wave patterns do no move left or right. Combining the two waves, changing
the sign in from of $\omega$ leads to the equation
\[
y'(x,t)=[2y_{m}\sin kx]\cos\omega t
\]


which describes a standing wave. In a standing wave, amplitude varies
with position. A standing wave can be set up by sending a stretched
string by allowing a traveling wave to be reflected from the far end
of the string so the wave travels through itself.


\subsubsection*{Standing Waves and Resonance}

A wave is a standing wave if it is at a resonant frequency, which
is referred to as resonance. If the string is oscillated at some other
frequency other than a resonant frequency, the wave is not a standing
wave. Resonant frequencies can be found by
\[
f=\frac{v}{\lambda}=n\frac{v}{2L}\, for\, n=1,2,3
\]


The fundamental mode of the first harmonic is the oscillation mode
with the lowest frequency. The second harmonic is the oscillation
mode with $n=2$, and so on. These oscillation modes are called the
harmonic series, and \emph{n }is the harmonic number of the \emph{n}th
harmonic. Resonance can occur in three dimensions.
\end{document}
