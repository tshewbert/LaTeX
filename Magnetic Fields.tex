%% LyX 2.1.1 created this file.  For more info, see http://www.lyx.org/.
%% Do not edit unless you really know what you are doing.
\documentclass[english]{article}
\usepackage[T1]{fontenc}
\usepackage[latin9]{inputenc}
\usepackage{amsmath}
\usepackage{graphicx}
\usepackage{esint}
\usepackage{babel}
\begin{document}

\subsection*{Chapter 28: Magnetic Fields}

\emph{Force by a magnetic field:}$\vec{F}=q\overrightarrow{v}\times\overrightarrow{B}$

\emph{Forces by an electric field}: $\overrightarrow{F}=q\overrightarrow{E}=q\frac{V}{d}$

\emph{Circular motion and magnetic fields}: $qvD=\frac{mv^{2}}{r}$

$r=\frac{mv}{qB}$

$f=\frac{\omega}{2\pi}=\frac{1}{T}=\frac{qB}{2\pi M}\Leftrightarrow T=\frac{2\pi}{Bq}$

\emph{Magnetic Field with a Current:}$\overrightarrow{F}=i\overrightarrow{L}\times\overrightarrow{B}$

\emph{Magnitude}: $F=iLB$

\emph{Torque:} $\tau=\overrightarrow{u}\times\overrightarrow{B}$

$\overrightarrow{u}=NiA$ where \emph{N }is the number of loops in
the coil.


\subsection*{Chapter 29: Magnetic Fields Due to Currents}

\emph{Biot-Savart Law:} The magnetic field set up by a current- carrying
conductor can be found from the Biot \textendash{} Savart law. This
law asserts that the contribution dB: to the field produced by a current-length
element i d:s at a point P located a distance r from the current element
is: $d\vec{B}=\frac{u_{0}}{4\pi}\frac{id\overrightarrow{s}\times\hat{r}}{r^{2}}$

Here r\^{ } is a unit vector that points from the element toward P.
The quantity m 0, called the permeability constant, has the value$u_{0}=4\pi\times10^{-7}T\cdot m/A$

\emph{Magnetic Field of a Long Straight Wire}: For a long straight
wire carrying a current i, the Biot \textendash{} Savart law gives,
for the magnitude of the magnetic field at a perpendicular distance
R from the wire: $B=\frac{u_{0}i}{2\pi R}$ (long straight wire)

\emph{Magnetic Field of a Circular Arc: }The magnitude of the magnetic
field at the center of a circular arc, of radius R and central angle
$\phi$(in radians) carrying current i is$B=\frac{u_{0}i\phi}{4\pi R}$
(at center of circular arc)

\emph{Force Between Parallel Currents: }Parallel wires carrying currents
in the same direction attract each other, whereas parallel wires carrying
currents in opposite directions repel each other. The magnitude of
the force on a length L of either wire is: $F_{ba}=i_{b}LB_{b}sin90^{\textdegree}=\frac{u_{0}Li_{a}i_{b}}{2\pi d}$


\subsection*{Chapter 30: Induction}

\emph{Magnetic Flux} through $\Phi_{B}$through an area A in a magnetic
field $\overrightarrow{B}$ is defined as $\Phi_{B}=\int\overrightarrow{B}\cdot d\overrightarrow{A}$,
where the integral is taken over the area. The SI unit of flux is
the weber, which is equal to 1 $T\cdot m$$^{2}$. If $\overrightarrow{B}$
is perpendicular to the area and uniform over it, it simplifies.

\emph{Faraday's Law of Inductions }states that if the magnetic flux
through and area bounded by a closed conducting loop changes with
time, a current and emf are produced in the loop, called induction.
$\varepsilon=\frac{d\Phi}{dt}$, and if it is inside a coil of closely
packed N turns, $\varepsilon=-N\frac{d\Phi}{dt}$

\emph{Lenz's Law }states that an induced current has the direction
such that the magnetic field caused by the current opposes the change
in magnetic flux that induces the current. Therefore, the induced
emf has the same direction as the induced current.

\includegraphics[scale=0.7]{/Users/TylerShewbert/Desktop/LenzsLaw}

\emph{Emf and the induced electric field }an emf is induced by magnetic
flux even if the loop through which the flux is flowing is not an
actual conductor. The changing magnetic field induces an electric
field $\overrightarrow{E}$ at every point of such a loop. The can
be described as $\varepsilon=\oint\overrightarrow{E\cdot}d\overrightarrow{s}$
which can be expanded to rewrite Faraday's Law$\varepsilon=\oint\overrightarrow{E\cdot}d\overrightarrow{s}=\frac{d\Phi}{dt}$

\emph{Inductors }are devices that can be used to produce a known magnetic
field in a specified region. If a current \emph{i }is established
through each of the \emph{N }windings of an inductor, a magnetic flux
$\varPhi_{B}$links those windings. Inductance L can be described
as: $L=\frac{N\varPhi_{B}}{i}$. The SI unit for inductance is the
henry, H, which is equal to $T\cdot\frac{m^{2}}{A}$. The inductance
per unit length near the middle of a long solenoid of a a cross-sectional
area \emph{A }and \emph{n }turns per until length is: $\frac{L}{l}=\mu n^{2}A$.

\emph{Self induction }can be caused when a current\emph{ i }in a coil
changes with time, causing an induced emf in the coil. This can be
described as: $\varepsilon_{L}=-L\frac{di}{dt}$. The direction of
the self-induced emf is found using Lenz's Law, for the self-induced
emf acts to oppose the change that produces it.

\emph{RL Circuits in Series:} If a constant emf is introduced into
a single loop circuit contain a resistance \emph{R }and an inductance
\emph{L, }then the \emph{current rises to equilibrium} $\frac{\varepsilon}{R}$
by \emph{$i=\frac{\varepsilon}{R}(1-e^{-t/(L/R)})$,} where (L/R)
is $\tau$ and is referred to as the inductive time constant. The
\emph{decay of current} can be described by the equation $i=i_{0}e^{-t/(L/R)}$

\emph{Magnetic Energy: }If an inductor \emph{L }carries a current
\emph{i,} the energy that the inductors magnetic field stores is described
by: $U_{b}=\frac{1}{2}Li^{2}$. If \emph{B} is the magnetic field
at any point then the density of that magnetic energy is $u_{b}=\frac{B^{2}}{2\mu}_{0}$

\emph{Mutual Induction }can occur when two coils near each other,
and a changing current is in either coil, can induce an emf in the
other, and can be described as $\varepsilon=-M\frac{di}{dt}$, where
\emph{M }is in henries and is the mutual inductance of two rates of
current and two emfs.
\end{document}
