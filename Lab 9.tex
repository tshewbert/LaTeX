%% LyX 2.1.1 created this file.  For more info, see http://www.lyx.org/.
%% Do not edit unless you really know what you are doing.
\documentclass[english]{article}
\usepackage[T1]{fontenc}
\usepackage[latin9]{inputenc}
\usepackage{textcomp}
\usepackage{amstext}
\usepackage{graphicx}

\makeatletter

%%%%%%%%%%%%%%%%%%%%%%%%%%%%%% LyX specific LaTeX commands.
%% Because html converters don't know tabularnewline
\providecommand{\tabularnewline}{\\}

\makeatother

\usepackage{babel}
\begin{document}
Physics 4BL\\
Wednesday 2-5p\\
\\
Tyler Shewbert\\
\\
Linlin Miao\\
Niral Bhsavar\\
Sabrina Tang\\
\\
Lab 9: e/m Lab\\

\begin{abstract}
The purpose of the e/m lab is to measure the ratio of \emph{e,} the
charge of an electron, to \emph{m,} the mass of an electron\emph{.
}ThisThis can be attained by measuring the path of an electron and
relating the current, voltage, radius and magnetic field to find a
value for e/m. The measures value of e/m was $2.62172\times10^{11}\pm28188784941$C/kg,
after q testing one of the trials out. The value before this trial
was q tested out was $2.63633\times10^{11}\pm33139039356$ C/kg. Several
qualitative observations of the electrons path were also taken in
Part two of the experiment reveling how magnetic fields affect the
path of the electron.
\end{abstract}

\subsubsection*{Procecure/Theory}

Experimental Procedure: Ref. CCSF Physics 4BL Lab Manual; J. Court,
R. King, D. Yee. 2013. Pgs 119-122.\\


\emph{Quantitative}: Measure the current, voltage, and radius of an
electron beam traveling through a Helmholtz coil. Take five measurements.
These will be plotted in a graph where the slope will be the value
e/m. Use the thickness of the electron beam for the uncertainty of
the radius value. The kinetic energy of an accelerated electron is
equal to the energy gained across a potential difference. From this,
the velocity of the electron can be found. The magnetic field the
Helmholtz coil produces causes the electron to be deflected into a
circular path with a centripetal acceleration, which enables the velocity
to be eliminated. B is given as a function of I. Current, I, voltage,
V, and the radius, r, can all be measured. The derivation is as follows,
starting with potential energy (PE) described by the charge of an
electron, \emph{e, }and potential difference, $\Delta V$:

\[
PE=e\Delta V
\]


\[
e\Delta V=\frac{1}{2}mv^{2}\:(1)
\]


The circular motion of the electron as it travels through the Helmholtz
coils can be described by:

\[
ma=F_{net}
\]
the net force, and

\[
F_{B}=eVB
\]
the force of the magnetic field
\[
a=\frac{v^{2}}{R}
\]


The circular acceleration

\[
m\frac{v^{2}}{R}=eVB\:(2)
\]


Equations (1) and (2) can be combined to form the equation:
\[
R^{2}=\frac{2\Delta V}{\frac{e}{m}B^{2}}
\]


Where R is the radius of the path of the electron as it i deflected,
$\Delta V$ is the voltage applied, $B=(6.60\times10^{-4}T/A)I$,
where \emph{I} is the current, \emph{e }is the charge of the electron,
and \emph{m }is the mass of the electron.

This can be plotted as:
\[
\frac{1}{R^{2}}\: versus\:\frac{B^{2}}{2\Delta V}
\]


Where \emph{R }is the dependent variable. This is a linear function
and the slope of the line produced is the value of \emph{e/m} as determined
by the collection of values of radius and voltage and the given value
of \emph{B} as a funtion of \emph{I}.

\emph{Qualitative}: Draw pictures of the deflected electron beam,
positive current of Helmholtz coils, and magnetic field due to the
Helmholtz coils Rotate the bulb to 45 degrees and record observations
Position a bar magnet to deflect the electron beam downwards.


\subsubsection*{Data}

The CODATA value of e/m is $(1.758820088\pm3900)\times10^{11}\text{\textpm}3900$
C/kg.

\begin{table}
\begin{centering}
\begin{tabular}{|l||l||l||l|}
\hline 
Data Collected from Anameter & Data Collected from Voltmeter & Calculated data of Electron Beam & Uncertainty based on thickness\tabularnewline
\hline 
Currrent (A) & Voltage (V) & Radius (cm) & Thickness (cm)\tabularnewline
\hline 
\hline 
1.280 & 99.9 & 3.940 & 0.212\tabularnewline
\hline 
\hline 
1.067 & 99.5 & 4.349 & 0.530\tabularnewline
\hline 
\hline 
1.129 & 99.5 & 4.470 & 0.087\tabularnewline
\hline 
\hline 
1.346 & 99.5 & 3.513 & 0.058\tabularnewline
\hline 
\hline 
1.571 & 99.5 & 2.821 & 0.306\tabularnewline
\hline 
\hline 
Data collected on Electron Beam &  &  & \tabularnewline
\hline 
\hline 
Diameter, i, (cm) & Diameter, f, (cm) & Thickness i, (cm) & Thickness, f, (cm)\tabularnewline
\hline 
\hline 
10.039 & 2.160 & 9.608 & 9.820\tabularnewline
\hline 
\hline 
2.530 & 11.228 & 11.576 & 11.046\tabularnewline
\hline 
\hline 
11.066 & 2.126 & 10.831 & 10.744\tabularnewline
\hline 
\hline 
9.521 & 2.496 & 9.670 & 9.612\tabularnewline
\hline 
\hline 
8.442 & 2.800 & 8.815 & 8.509\tabularnewline
\hline 
\hline 
 &  &  & \tabularnewline
\hline 
\hline 
B (T) ($6.60\times10^{-4}\times$I) & Trial & $B^{2}$/2V (T/V) & 1/$R^{2}$ (m)\tabularnewline
\hline 
\hline 
0.0008448 & 1 & 3.57201E-09 & 644.3440023\tabularnewline
\hline 
\hline 
0.00070422 & 2 & 2.49209E-09 & 528.7144557\tabularnewline
\hline 
\hline 
0.00074514 & 3 & 2.79012E-09 & 500.4779564\tabularnewline
\hline 
\hline 
0.00088836 & 4 & 3.96575E-09 & 810.5267157\tabularnewline
\hline 
\hline 
0.00103686 & 5 & 5.40241E-09 & 1256.590661\tabularnewline
\hline 
\hline 
Data from Linest Functions &  &  & \tabularnewline
\hline 
\hline 
 & Value of e/m without Trial 1 Data (C/k & Intercept & \tabularnewline
\hline 
\hline 
e/m & $2.62172\times10^{11}$ & -186.1507733 & \tabularnewline
\hline 
\hline 
Uncertainty & 28188784941 & 108.1770087 & \tabularnewline
\hline 
\hline 
 &  &  & \tabularnewline
\hline 
\hline 
 & Orignal Value of e/m (C/kg) & Intercept & \tabularnewline
\hline 
\hline 
e/m & $2.63633\times10^{11}$ & -212.6725964 & \tabularnewline
\hline 
\hline 
Uncertainty & 33139039356 & 125.4632247 & \tabularnewline
\hline 
\end{tabular}
\par\end{centering}

\protect\caption{Data from Excel Spreadsheet}
\end{table}



\subsubsection*{Calculations}


\subsubsection*{Data Analysis}

Upon entering the data into the spreadsheet and graphing the data,
it was found that one of the data points was found to not fit linearly
on the graph. After q testing this trial of data out, a new graph
using only four data points was achieved with a linear slope. Therefore,
there are two linear slopes, of values $2.63633\times10^{11}\pm33139039356$
C/kg. and $2.62172\times10^{11}\pm28188784941$C/kg respectively,
the uncertainties were calculated using the LINEST function uncertainties.
The expected value of e/m is $1.76\times10^{11}$. The uncetainties
from the annameter, the volt meter, the thickness of the electron
beam, all which \emph{have not} been accounted for in the linest function.
The discrepancy test is based on the second value without trial one
and the CODOC value of uncertainty is:
\[
\mbox{}
\]


For the second part, see the figures in the data section. In Figure
, it can be seen that the magnetic field in out of the page, because
if at an instantaneous point on the electron beam, the value of v
is tangent to the beam, and F is pointing towards the center. Using
the right hand rule it can be determined that B is coming out of the
page after reversing the hand due to the negative charge of the electron.
\begin{figure}
\begin{centering}
\includegraphics{attachments/a}
\par\end{centering}

\protect\caption{Lab Notebook drawing for Question A}
\end{figure}


For part B, the electron path turns into a helix as the bulb is rotated
to a 45 degree angle in relation to the Helmholtz coils. It can be
concluded that as the bulb is rotated, the electron path turns into
a helix as the magnetic field affects the relationship of the electrons
as the exit the bottom of the bulb at a different angle.

\begin{figure}
\centering{}\includegraphics{attachments/b}\protect\caption{Lab Notebook drawing for Question B}
\end{figure}


For part C, it is observed that when the bar magnet is place next
to the electron beam with the south pole towards the beam, that the
electron beam is deflected downward. See Figure:

\begin{figure}
\centering{}\includegraphics{attachments/c}\protect\caption{Lab Notebook drawing for Question }
\end{figure}



\subsubsection*{Questions}

The questions in part two were answered in the Data Analysis section.


\subsubsection*{Conclusion}

In conclusion the e/m value was determined to be: . It the discrepancy
test. The qualitative data showed that the magnetic field was pointing
towards the telescopes as the electron beams were observed, and that
if the bulb is rotated from its perpendicular angle the potential
for errors in the calculations can increase.
\end{document}
