
\documentclass[english]{article}
\usepackage[T1]{fontenc}
\usepackage[latin9]{inputenc}
\usepackage{babel}
\begin{document}

\title{Sound Waves}

\maketitle

\subsubsection*{What are sound waves? }

Sound waves can be described as any longitudinal wave, which is a
wave that involve oscillations parallel to the direction of wave travel.

When dealing with sound waves, we can imagine a point source from
which sound waves radiate in all directions. \textbf{Wavefronts} and
rays indicate the direction of travel. Wavefronts mathematical similar
to contour lines, the oscillations due to the wave have the same value.
\textbf{Rays} are the gradients of the wavefronts. Near a source,
sound waves are spherical, far from a source the begin to to behave
in a planar fashion.


\subsubsection*{The Speed of Sound}

We can generalize the formula used in the previous section on waves
to describe the speed of sound as:

\[
v=\sqrt{\frac{\tau}{\mu}}=\sqrt{\frac{elastic\, property}{intertial\, property}}
\]


As a sound wave travels through air, the potential energy associated
with the wave is associated with the compressions and expansions of
the elements of the air. \textbf{Bulk modulus} \emph{B }is used to
describe the property that determines the extent to which an element
of a medium change in volume when the pressure changes: 

\[
B=-\frac{\Delta p}{\Delta V/V}
\]


$\Delta V/V$ is the fraction change in volume produced by a change
pressure, \emph{p. }The SI unit for pressure is the \emph{pascal,
}Pa. This is the elastic property described above. Using this, and
density$\rho$ as the inertial property, we can rewrite the equation
as:
\[
v=\sqrt{\frac{B}{\rho}}
\]



\subsubsection*{Traveling Sound Waves}

We can examine the displacements and pressure variations of a sinusoidal
sound wave if we consider that if we place a piston at the end of
a tube. The motion of the piston will causes compression and expansion
in the tube. This will change the air pressure and travel along the
tube as a sound wave. A thin element of air, $\Delta x$ will oscillate
left and right in simple harmonic motion. Because of this, the oscillation
of each air element oscillate longitudinally rather than transversely
as in a string wave. This can be described by:
\[
s(x,t)=s_{m}\cos(kx-\omega t)
\]


In this equation, the value $s_{m}$is the displacement amplitude,
the maximum displacement of the air element to either side of its
equilibrium position. As the wave moves, the air pressure at any point
\emph{x }is:
\[
\Delta p(x,t)=\Delta p_{m}\sin(kx-\omega t)
\]
A negative $\Delta p$ indicates that the air is expanding, and a
positive value indicates it is compressing. $\Delta p$ is the pressure
amplitude, which is the maximum increase or decrease in pressure due
to the wave. The pressure amplitude and displacement amplitude can
be related by the equation:
\[
\Delta p=(v\rho\omega)s_{m}
\]



\subsubsection*{Interference}

Sound waves also undergo interference. The phase difference $\phi$is
dependent on the path length difference, $\Delta P$. These can be
related by the formula
\[
\phi=\frac{\Delta L}{\lambda}2\pi
\]


Fully constructive interference occurs when $\phi=n(2\pi)$, and fully
destructive interference occurs when $\phi=(2n+1)\pi$


\subsubsection*{Intensity and Sound Level}

Intensity of a sound wave at a surface is the average rate at which
the wave transfers energy onto or through the surface. This is described
by the equation:
\[
I=\frac{P}{A}
\]


where \emph{P }is power of the sound wave and \emph{A }is the surface
area which the power of the sound wave is being transferred to. This
intensity can be related to the displacement amplitude by the equation:
\[
I=\frac{1}{2}\rho v\omega^{2}s_{m}^{2}
\]


The variation of intensity with distance can be complex. Often, real
sources of sound are only transmitting in one direction, and echoes
are also created overlapping the direct sound waves. To simplify this,
we can ignore echoes and assume that the source emits sound isotropically,
meaning with equal intensity in all directions. Using this assumptions,
we can say that the point source \emph{S }is at the center of an imaginary
sphere radius \emph{r. }Therefore, all the energy must pass through
the surface of the sphere. Intensity at the sphere can then be described
by the equation:
\[
I=\frac{P_{s}}{4\pi r^{2}}
\]


Since $4\pi r^{2}$ is the area of the sphere, it can be concluded
that the intensity in this instance decrease with $r^{2}$.

Humans can hear over a large range of intensities. The humans ear
displacement amplitude ranges from $10^{-5}$ to $10^{-11}$, from
loudest to quietest sound. Often, instead of talking about sound intensity,
we refer to sound level, which is defined as:

\[
\beta=(10dB)\log\frac{I}{I_{o}}
\]


The unit of sound level is the decibel, named after Alexander Graham
Bell. $I_{0}$is the standard reference intensity, which is equal
to $10^{-12}W/m^{2}$, which is near the lower end of the human hearing
limit.


\subsubsection*{Sources of Musical Sound}

Musical sounds can be set up by multiple oscillating bodies. If one
sets up a standing wave that corresponds to the resonance frequency
of the string, then a large, sustained amplitude can be achieved,
pushing back and forth against the surrounding air, and therefore
creating a sound with the same frequency as the string. Similarly,
air in pipe can be set up like the wave on a string, causing a resonance
within the pipe that oscillates with a large, sustained amplitude.
In sound, the standing waves number \emph{n }is referred to as the
harmonic number. Using the formula for the standing wave, we can develop
a resonance frequency for a pipe with two open ends
\[
f=\frac{v}{\lambda}=\frac{nv}{2L}
\]


For a pipe with only one open end, all the harmonic numbers must have
an \emph{n }value that is an odd integer. A musical instrument's length
determines the range of frequencies over which the instrument can
function. The smaller the length, the higher the frequencies.


\subsubsection*{Beats}

While listening to two similar frequencies simultaneously we cannot
tell one from the other, and this causes us to hear a sound whose
frequency is an average of the two frequencies. We will hear a variation
in intensity of this sound, wavering beats that seem to increase and
decrease. The frequency of the beat can be described by the equation
\[
f_{beat}=f_{1}-f_{2}
\]



\subsubsection*{The Doppler Effect}

Motion related frequency changes are called the doppler effect, such
as the the change of the frequency of a siren as you approach it in
a moving vehicle. The Austrian physicist Johann Christian Doppler
originally proposed the effect in 1842, and tested experimentally
in Holland in 1845. The Doppler effect holds true for sound waves
and electromagnetic waves. Since we are dealing only with sound waves
in this section, we will consider the speed of the source of sound
by the detector was relative to that body of air. If the the sound
and detector are moving toward or directly away from each other, the
emitted frequency \emph{f }and detected frequency \emph{f' }can be
related by the formula
\[
f'=f\frac{v\pm v_{D}}{v\pm v_{S}}
\]


where \emph{v }is the speed of sound through air, $v_{D}$ is the
detector's speed relative to the air, and $v_{S}$ is the source's
speed relative to the air. When the motion of the detector or the
source is toward the other, the sign on its speed must give and upward
shift in frequency, and when the motion is away from the other, the
sign on it must give a downward shift in frequency. Towards means
shift up and away means shift down. See Derivation of the Doppler
Effect.


\subsubsection*{Supersonic Speeds and Shock Waves}

For supersonic speeds, which means that the source is moving so fast
that it is keeping pace with its own spherical wavefronts, the equations
of the Doppler Effect no longer apply. The wavefronts begin to bunch
together and form what is referred to as a \emph{mach cone}. There
is a shockwave along the surface of this cone, because the bunching
of wavefronts causes the air pressure to abruptly rise and fall. The
mach cone angle is shown by the equation
\[
\sin\theta=\frac{vt}{v_{S}}=\frac{v}{v_{S}}
\]


$v_{S}/v$ is the Mach number. 
\end{document}
