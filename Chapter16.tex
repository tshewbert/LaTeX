
\documentclass[english]{article}
\usepackage[T1]{fontenc}
\usepackage[latin9]{inputenc}
\usepackage{babel}
\begin{document}

\title{Waves, Part One}

\maketitle

\subsubsection*{Types of Waves}
\begin{enumerate}
\item Mechanical Waves: These are the most common waves and include water
waves, seismic waves and sound waves. They are governed by Newton's
Laws and much exist within a material medium.
\item Electromagnetic Waves: These waves include x-rays, ultraviolet light,
radio, microwaves and radar. Thes waves do not require a medium to
travel through to exist. EM waves travel through a vacuum at the speed
of light.
\item Matter Waves: These waves are associated with protons, neutrons, electron,
and other fundamental particles.
\end{enumerate}

\subsubsection*{Transverse and Longitudinal Waves}

A wave can be sent along a stretched, taut string is a a simple form
of a mechanical wave. A pulse will travel along the string. As this
pulse distorts the string's shape the pulse moves along the string
at some velocity $\overrightarrow{v}.$ If a continuous wave travels
down the string at $\overrightarrow{v}$ in continuous harmonic motion,
the motion that causes the pulse can be described by a sinusodial
function of time. In an ideal string, no friction-like forces within
the string will cause the wave to die out as it travels along. In
an ideal string, it is also assumed that we need not consider the
wave rebounding from the opposite end from which it began.

Monitoring wave forms is one way to study waves. Another option is
to monitor an element of string as it oscillates up and down as a
wave passes it. The displacemet of the the up and down oscillations
in ortanthogonal to the direction of travel of the wave. This motion
is said to be \emph{transverse}, and the wave is refered to as a \emph{transverse
wave.} 

For a \emph{longitudal wave}, the example of pushing and pulling on
a piston filled with air can be used. The motion of the air is parallel
to the direction of the wave's travel is referred to as \emph{longitudal.}

Both of these types of waves are \emph{travelling waves}, since they
travel from one point to another. It is the wave that moves from end
to end, not the material through which the wave moves.


\subsubsection*{Wavelength and Frequency}

To describe the wave on a string a function that gives the shape of
the wave is needed. This means a relation in the form:

\[
y=h(x,t)
\]


Where \emph{y }is the transverse displacement of any string element
described by the function \emph{h, }where \emph{t }is time and \emph{x
}is the position of the element along the string. In general the shape
is sinosodial, then the sine or cosine function can be used to describe
the shape. Here the sine function will be used:

\[
y(x,t)=y_{m}\sin(kx-\omega t)
\]


This equation is written in terms of \emph{x,} meaning that it can
be used to find the displacements of all the elements with respect
to time, \emph{t. }

$y_{m}$is the amplitude of a wave, the maximum displacement of the
wave elements from their equilibrium positions as the wave passes
through them. Since it is magnitude, amplitude can always be considered
a positive quantity.

The phase is described by $kx-\omega t$. As the waves sweeps through
an element of the string linearly through a particularly position,
the phases changes with respect to time \emph{t. }Therefore, sine
also changes, alternating between 1 and -1. Theses max and min values
of the sine function coresspond to the peaks and vallleys of $y_{m}$.
The amplitude of the wave determines the extremes of the elements
displacement.


\subsubsection*{Wavelength and Angular Wave Number}

The wavelength, $\lambda$,of a wave is parrallel to the direction
of the wave's travel and describes the distances between the shape
of the waves. The equation that describes the wavelength can be written:
\[
y(x,0)=y_{m}\sin kx
\]


The displacement \emph{y} at both ends of the wavelength must be the
same, so $x=x_{1}$and $x=x_{1}+\lambda$. This allows the previous
equation to be rewritten as:

\begin{eqnarray*}
y_{m} & = & \sin k(x_{1}+\lambda)\\
 & = & \sin(kx_{1}+k\lambda)
\end{eqnarray*}


Since a sine function repeats itself every $2\pi$, $k\lambda=2\pi$
or $k=\frac{2\pi}{\lambda}$, where \emph{k }is the angular wave number
of the wave, and its units are rad/m, or the inverse meter.


\subsubsection*{Period, Angular Frequency and Frequency}

If a single element of the string is monitored, it will be seen that
at that position \emph{x }the string moves up and down in simple harmonic
motion. If $x=0$, then $y(0,t)=y_{m}\sin(k(0)-\omega t)$ and then
$-y_{m}\sin(\omega t)$. If this is graphed, it will show the displacement
versus time, not the shape of the wave.

A \emph{period} of oscillation \emph{T }is defined at the time any
string element takes to move through one full oscillation. To find
the angular frequency $\omega$ or period \emph{T, }we use $-y_{m}\sin(\omega t)$:

\begin{eqnarray*}
-y_{m}\sin(\omega t) & = & -y_{m}\sin\omega(t_{1}+T)\\
 & = & -y_{m}\sin(\omega t_{1}+\omega T)
\end{eqnarray*}


This is only true if $2\pi=\omega T$. The can be written to describe
\emph{angular frequncy:}

\[
\omega=\frac{2\pi}{T}
\]


The SI unit of angular frequency is radians per second.

The frequency \emph{f }of a wave is defined as $1/T$ and is related
to angular frequency by:
\[
f=\frac{1}{T}=\frac{\omega}{2\pi}
\]



\subsubsection*{Phase Constant}

The phase constant $\phi$ can be used to generalize the equation
$y(x,t)=y_{m}\sin(kx-\omega t)$ by transforming it to $y(x,t)=y_{m}\sin(kx-\omega t+\phi)$.
The value of $\phi$ can be chosen so the function gives a different
displacement and slope at $x=0$ and $t=0$. Basically, the function
is shifted by the value of $\phi$.


\subsubsection*{The Speed of a Travelling Wave}

If a wave is travelling in the positve \emph{x }direction over a period
of time, this can be described by $\Delta x/\Delta t$. At an instantaneous
point, this can be written in the differntial form $\frac{dx}{dt}$,
which is the wave speed\emph{ v.}

If the displacement \emph{y }remains the same, then $kx-wt$ is a
constant, but \emph{x }and \emph{t }are both changing, which means
that as time increases, so does displacment to keep the argument constant,
confriming that the wave pattern in moving in a positive direction.

To find the wave speed, we can take the dervative of $kx-wt$:

\begin{eqnarray*}
k\frac{dx}{dt}-\omega & = & 0\\
\frac{dx}{dt}= & v & =\frac{\omega}{k}
\end{eqnarray*}


Asssuming that $k=2\pi/\lambda$and $\omega=2\pi/T$, these equations
can be rewritten as:

\[
v=\frac{\omega}{k}=\frac{\lambda}{T}=\lambda f
\]


Which describes wave speed travelling in the positive direction, \emph{x.}

For a wave speed in the negative \emph{x }direction:

\[
\frac{dx}{dt}=-\frac{\omega}{k}
\]



\subsubsection*{Wave Speed on a Stretched String}
\end{document}
